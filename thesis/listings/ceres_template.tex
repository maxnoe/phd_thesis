#                              General
# -------------------------------------------------------------------------
# Use this if you want to setup the logging stream for the jobs
# (overwrites command line options)
# -------------------------------------------------------------------------
MLog.VerbosityLevel: 4
MLog.DebugLevel:     0
MLog.NoColors:       no

#                                   Ceres
# -------------------------------------------------------------------------
# Configure Eventloop
# -------------------------------------------------------------------------
Overwrite: yes


# -------------------------------------------------------------------------
# Use this to setup binnings. For more details see: MBinning::ReadEnv
# -------------------------------------------------------------------------
BinningImpact.Raw:  40 0 1000
BinningTrigPos.Raw: 300 -25 275


# -------------------------------------------------------------------------
# Initialize random number generator (see MJob::InitRandomNumberGenerator)
# -------------------------------------------------------------------------
RandomNumberGenerator: TRandom3
RandomNumberSeedValue: 1

# -------------------------------------------------------------------------
# Some setup for the atmosphere. The default should be well suited.
# -------------------------------------------------------------------------
MSimAtmosphere.FileAerosols: {{ resource_directory }}/atmosphere-aerosols.txt
MSimAtmosphere.FileOzone:    {{ resource_directory }}/atmosphere-ozone.txt


# -------------------------------------------------------------------------
# Here you can control the poiting of the telescope. To switch on
# off-target observations set a value for the distance !=0 [deg].
# For details see MSimPointingPos
# -------------------------------------------------------------------------

MSimPointingPos.OffTargetDistance: -{{ run.off_target_distance }}

MSimPointingPos.OffTargetDistance: {{ run.off_target_distance }}



# -------------------------------------------------------------------------
# Setup the reflector and camera geometry
# -------------------------------------------------------------------------
Reflector.Constructor: MReflector

# For the file definition see MReflector::ReadFile

Reflector.FileName: {{ resource_directory }}/reflector.txt
Reflector.SetSigmaPSF: {{ settings.psf_sigma }}
MGeomCam.Constructor:   MGeomCamFACT();

#MSimBundlePhotons.FileName: resmc/fact/dwarf-fact.txt

# Set the APD type (1: 30x30 <default>, 2: 60x60, 3:60x60(ct=15%))
MSimAPD.Type:                   0
MSimAPD.NumCells:               60
MSimAPD.DeadTime:               {{ settings.apd_dead_time }}
MSimAPD.RecoveryTime:           {{ settings.apd_recovery_time }}
MSimAPD.CrosstalkCoefficient:   {{ settings.apd_cross_talk }}
MSimAPD.AfterpulseProb1:        {{ settings.apd_afterpulse_probability_1 }}
MSimAPD.AfterpulseProb2:        {{ settings.apd_afterpulse_probability_2 }}

MSimExcessNoise.ExcessNoise:    {{ settings.excess_noise }}

# -------------------------------------------------------------------------
# Setup the absorption, conversion efficiency and angular acceptance
# -------------------------------------------------------------------------
MirrorReflectivity.FileName: {{ resource_directory }}/mirror-reflectivity.txt
PhotonDetectionEfficiency.FileName: {{ resource_directory }}/pde.txt
ConesAngularAcceptance.FileName: {{ resource_directory }}/cone-angular-acceptance.txt
ConesTransmission.FileName: {{ resource_directory }}/cone-transmission.txt

AdditionalPhotonAcceptance.Function.Name: {{ settings.additional_photon_acceptance }}
AdditionalPhotonAcceptance.Function.Npx:  100
AdditionalPhotonAcceptance.Function.Xmin: 290
AdditionalPhotonAcceptance.Function.Xmax: 900

# -------------------------------------------------------------------------
# Setup what in MMCS would be called "additional spot size"
# -------------------------------------------------------------------------
#MSimPSF.Sigma: -1
#MSimReflector.DetectorMargin: 0

# -------------------------------------------------------------------------
# Setup the dark counts (FrequencyFixed) and the NSB noise per cm^2
# -------------------------------------------------------------------------
MSimRandomPhotons.FrequencyFixed: {{ settings.dark_count_rate }}

MSimRandomPhotons.FrequencyNSB: {{ run.nsb_rate }}

MSimRandomPhotons.FileNameNSB:  {{ resource_directory }}/nsb.txt



# -------------------------------------------------------------------------
# Setup the trigger
# -------------------------------------------------------------------------
# This line could be omitted but then the discriminator would be
# evaluated for all pixels not just for the pixels which are
# later "connected" in the trigger (used in the coincidence map)
MSimTrigger.FileNameRouteAC: {{ resource_directory }}/route-ac.txt
MSimTrigger.DiscriminatorThreshold: {{ settings.discriminator_threshold }}

MSimTrigger.CableDelay: 21.0
MSimTrigger.CableDamping: -0.96
MSimTrigger.CoincidenceTime: 0.5

# Every Pixel(!) should see the same signal independant of its size
MSimCalibrationSignal.NumPhotons: 24
MSimCalibrationSignal.NumEvents:  1000

IntendedPulsePos.Val: 26

# -------------------------------------------------------------------------
# Setup the FADC
# -------------------------------------------------------------------------

MRawRunHeader.SamplingFrequency: 2000
MRawRunHeader.NumSamples:        300
MRawRunHeader.NumBytesPerSample: 2
MRawRunHeader.FadcResolution:    12

MSimCamera.DefaultOffset:       -1850.0
MSimCamera.DefaultNoise:        2.8125
MSimCamera.DefaultGain:         22.553


# Value for the fudgefactor in the calculation of the accoupling:
MSimCamera.ACFudgeFactor: 0.3136
MSimCamera.ACTimeConstant: 20


# The number of sampling points is almost irrelevant because they
# are equidistant, i.e. calculated and no search is necessary.
# Nevertheless, you must make sure that there are enough points
# to sample the function accuratly enough.
# Attention: x in the function is given in slices, so if you change the sampling
# frequency you have to change also this function
PulseShape.Function.Name:  {{ settings.pulse_shape_function }}
PulseShape.Function.Npx:   310
PulseShape.Function.Xmin:  -10
PulseShape.Function.Xmax:  300



# -------------------------------------------------------------------------
# This is a cut executed after the calculation of the image parameters
# -------------------------------------------------------------------------

Cut.Inverted: yes
Cut.Condition: MHillas.fSize>10.0


# Does not trigger anyway
ContEmpty3.Condition: MPhotonEvent.GetNumPhotons<10

MFixTimeOffset.FileName: {{ resource_directory }}/pixel-delays.csv
ResidualTimeSpread.Val: {{ settings.residual_time_spread }}
GapdTimeJitter.Val: {{ settings.gapd_time_jitter }}

# last line comment
