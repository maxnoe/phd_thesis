\section{Monte Carlo Production Software}\label{sec:mopro}


To run the large number of tasks required for the Monte Carlo simulations,
a software library and a set of executable programs were developed,
with the following requirements identified:

\begin{itemize}[noitemsep]
  \item Templating for the configuration files of the simulation programs,
    so that job parameters can be stored centrally.
  \item Tracking of job status, input and output files.
  \item Automatic installation and compilation of the needed software.
  \item With the above, also the possibility to use several compiled versions
    of \gls{corsika} and \texttt{ceres} to enable comparisons.
  \item Possibility to run computation tasks on multiple backends,
    at least on the \texttt{SLURM}~\cite{slurm} cluster engine and locally using multiple CPU cores.
  \item Reproducibility of jobs.
  \item Central job queue optionally with multiple consumers.
  \item Collection and compression of simulation results.
  \item Possibility to use multiple database engines, at least
    \texttt{sqlite} for easy, local processing and testing and a
    fully grown relational database like MySQL or PostgreSQL for larger
    productions.
\end{itemize}

The software called \mopro~\cite{mopro3} was implemented in \texttt{python}.
Two similar projects existed before, but were centered around the
older \gls{mmcs} and had several shortcomings with respect to the requirements
listed above, lacking support for automatic installation of software,
multiple consumers, running jobs locally and on multiple sites.

The job configuration and simulation settings are stored in a relational
database, using the \texttt{peewee}\cite{peewee} Object Relational Mapper.
The relational mapper allows access to the database from \texttt{python} as objects
and supports all major database systems.

Templating of the configuration files is done using the \texttt{jinja2} templating
library, an example of a \gls{corsika} input card template can be found in Appendix~\ref{apx:corsika_input}, the \ceres{} template in Appendix~\ref{apx:ceres-template}.
The expressions in double-curly braces are evaluated and replaced for each
database entry before giving it as input to \gls{corsika} and \ceres{}, respectively.

Two computation backends are currently implemented.
First, jobs can  be submitted
on a cluster using \texttt{SLURM}, which is used for job management
at the LIDO3 cluster in Dortmund and the LESTA cluster in Geneva, where \gls{fact}'s data
is stored.
Jobs can also be processed using local worker processes, which eases development
and running small simulation productions.
Other backends can be added by implementing a small number of required methods, e.\,g.\ for submitting, canceling and getting the status of jobs.

The main program queries the database in certain intervals for pending jobs,
and if possible submits them for computation to the workers.
An extra thread in the main program uses a \texttt{ØMQ}~\cite{zeroMQ} server socket
to receive job status updates which are then persisted to the database.

The simulation programs \gls{corsika} and \ceres{} are wrapped in small scripts to
copy input files to the local working node, handle failures and communicate
status transitions to the main program as mentioned above.

When a new job is submitted to the workers,
the main program makes sure all necessary software is installed, creates
the required directories for output and log files and
creates the configuration files from the templates and parameters stored
in the database.

As any \ceres{} job depends on the successful \gls{corsika} job it uses 
as input, \ceres{} jobs are started when their corresponding \gls{corsika}
job finishes successfully and it was computed at the same location.

\texttt{mopro3} itself is configured using a \texttt{yaml} file.
The settings include credentials for downloading \gls{corsika},
the database configuration, 
which temporary directory to use for job processing
and options for the workers like which queues to use on a \texttt{SLURM} cluster
or how many CPUs to use in parallel for local processing.

\subsection{Database Structure}

The \mopro{} database contains five tables,
one describing general settings for \gls{corsika},
one for each individual \gls{corsika} run, the same two tables for the
\ceres{} jobs and one table describing the job states.

For the \gls{corsika} settings and runs, a compromise between flexibility and
complexity of the database and implementation was made.
\gls{corsika} has both compile time and execution time options.
The compile time options are normally set interactively by executing
the \gls{coconut} program.
This creates a header file \texttt{config.h} defining preprocessor variables that result in conditionally
compiling the corresponding code. 
This is used among other things for the interaction models, thus the same \gls{corsika}
installation cannot be used with different interaction models.
Instead of implementing all possible compile time options in
the \texttt{CorsikaSettings} table and generating the \texttt{config.h} file
at installation time, the user has to generate the file using \gls{coconut} and 
the full file is stored in the database and used to compile \gls{corsika}.
This also has the advantage of being rather agnostic about the version of \gls{corsika}.

A similar approach is taken for the large number of runtime 
configuration options \gls{corsika} supports, which in turn also depend on
the compile time configuration.
Only the most fundamental settings most likely to change
between runs of the same kind are stored in the database.
Other parameters have to be changed via the input card template in the \texttt{CorsikaSettings} table.
This also has the advantage that they are very unlikely to change with a new release of \gls{corsika}, as all these options are very fundamental,
improving maintainability of \texttt{mopro3}.
In most cases, no changes should be required when a new minor version update of \gls{corsika} is released.

The \texttt{Status} table stores all possible states a run can reside in.
All states and their possible state changes are shown in \autoref{fig:states}.
\begin{figure}
  \centering
  \begin{tikzpicture}
    \tikzset{change/.style={font=\footnotesize, text width=3cm, color=gray, midway}}
    \node[fill=blue!20, font=\ttfamily] (created) {\strut created};
    \node[fill=blue!20, font=\ttfamily, right=1cm of created.east] (queued) {\strut queued};
    \node[fill=blue!20, font=\ttfamily, right=1cm of queued.east] (running) {\strut running};
    \node[fill=green!20, font=\ttfamily, right=3cm of running.east] (success) {\strut success};
    \node[fill=red!20, font=\ttfamily, below right=1cm and 3cm of running.east] (failed) {\strut failed};
    \node[fill=red!20, font=\ttfamily, above right=1cm and 3cm of running.east] (walltime) {\strut walltime\_exceeded};

    \draw[thick, ->] (created.east) to[out=30, in=150] node[change, above=0.25cm, align=center] {Job is submitted to a cluster} (queued.west);

    \draw[thick, ->] (queued.east) to[out=-30, in=210] node[change, below=0.25cm, align=center] {Job starts executing} (running.west);

    \draw[thick, ->] (running.east) to node[change, above, align=center] {Job finished} (success.west);

    \draw[thick, ->] (running.east) to[out=-45, in=180] node[change, below left, align=center] {An error occurred} (failed.west);

    \draw[thick, ->] (running.east) to[out=45, in=180] node[change, above left, align=right] {Job not finished in walltime} (walltime.west);
  \end{tikzpicture}
  \caption{%
    Possible state transitions for the \texttt{mopro3} processing.
    Blue indicates unfinished jobs, green successfully finished jobs and red failure modes.
  }\label{fig:states}
\end{figure}
