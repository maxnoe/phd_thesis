\chapter{Monte Carlo Simulation of FACT Events}\label{chp:simulation}

To evaluate the performance of an instrument design before it is build
or to generate labeled data for the training of statistical models,
the possibility of simulating the data of instruments is of crucial importance.
As there are no artificial sources of gamma rays and cosmic rays at the energies
measured by \glspl{iact}, extensive simulations are
necessary to be able to reconstruct the events observed with these telescopes.

Simulation of events for \glspl{iact} is usually done in two steps:
First, the \gls{eas} is simulated which produces the Cherenkov light
on the ground level at the telescope as output.
Second, from the simulated Cherenkov photons, the detector response
of the instrument is calculated.

Due to the stochastic nature of the involved processes, the Monte Carlo
method is the perfect fit to generate artificial \gls{iact} events with known
initial conditions.
Very few initial parameters result in an incalculable large number of possible outcomes.
Simulating a single air shower requires drawing several million pseudo random numbers.

For small instruments like \gls{fact}, the simulation of the \gls{eas} is
usually much more computationally intensive than the simulation of the detector response.

\section{Simulation of Air Showers Using CORSIKA}\label{sec:corsika}

The development of \gls{corsika}~\cite{corsika} initially began in the late 1980s by merging
three precursor code bases into a single software to simulate \gls{eas} for
the KASCADE~\cite{kascade} cosmic ray experiment.
It was later extended to also be able to calculate Cherenkov light production,
currently with the \texttt{iact/atmo} extension \cite{iactatmo}.

\gls{corsika} simulates all particle interactions from the moment
a particle enters Earth's atmosphere.
This requires knowledge of the cross-sections for all possible processes,
especially for very high energy interactions in forward direction and for
air molecules.
These regimes are generally not accessible by collider experiments and
are thus extrapolated from the available data at lower energies, larger
scattering angles and other nuclei~\cite{epos-lhc}.
Improvements to the cross-section calculations after new experimental data
is published is one of the main improvements between \gls{corsika} versions.

To produce simulations with \gls{corsika}, a large number of configuration options
have to be considered, both at compile and run time.
These need to be adapted for the specific experiments and several libraries
can be chosen for the calculation of the particle interactions.

\gls{corsika} is written in \texttt{FORTRAN 77},
while the \texttt{iact/atmo} extension is written in \texttt{C}.
Unfortunately, \gls{corsika} is not Free and Open Source Software.
However, access is usually granted for anyone interested in using the software.
There is an ongoing effort to create the next major version of \gls{corsika} 
using more modern technologies in a complete redesign,
openly developed by the community and published under an open source license.
However, this is not yet finished as of early 2020~\cite{corsika8}.

In this work, new simulations were produced using \gls{corsika}, version \texttt{7.69} released in December 2018 of \gls{corsika}.
These are compared to simulations that were previously created 
by the \gls{fact}-Collaboration
using a customized version of \gls{corsika} called \gls{mmcs}, which
was based on version \texttt{6.5} \cite{mmcs}, released in 2006.

\subsection{Interaction Models}\label{sec:models}

The particle interaction cross sections are not calculated by \gls{corsika} itself
but by externally contributed libraries.
When simulating Cherenkov light, 
the electromagnetic interactions are simulated using \gls{egs4}~\cite{egs4}.

For the hadronic interactions, several libraries are available for use
with \gls{corsika}.
As these libraries have their own energy domains, \gls{corsika} splits 
calculations into two energy regimes each handled by a different library with
a hard transition at \SI{80}{\GeV}~\cite{corsika_manual_769}.
Choosing the libraries for the different cross section calculations is a compile time option.

There are three possible choices for the hadronic interactions below an energy of
\SI{80}{\GeV} in \gls{corsika} 
for both versions \texttt{7.69} and \texttt{6.5}: \gls{gheisha}, \gls{fluka} and \gls{urqmd}~\cite{corsika_manual_769}.
According to a comparison of the three models in \cite{model_comp_auger}, \gls{gheisha} fails to reproduce measured data when used with \gls{corsika} to simulate air showers.
It was thus not considered for the simulation of \gls{fact} events.
\gls{fluka}~\cite{fluka} is only provided as compiled binary to registered users with a very restrictive
license~\cite{fluka-license}.
These binaries even have a \enquote{shelf life} date compiled into them,
after which the software will stop to work.
As this is all but prohibiting the reproducibility of the simulations, 
\gls{fluka} should not be used for a new set of simulations.
However, for a long time it was considered to be the best model~\cite{model_comp_auger} and it was used
to produce the simulations with \gls{mmcs} \texttt{6.5}.
\gls{urqmd}~\cite{urqmd}, the third option, will be explored as an alternative to 
\gls{fluka}.

The possible choices for the hadronic interaction models above \SI{80}{\GeV} are more numerous,
with some of the options being combinations of the other models.
The \gls{mmcs} \texttt{6.5} production used \gls{qgsjet}.
With the advent of \gls{LHC} measurements, many libraries were updated to reproduce the 
now available data at higher energies and for lead collisions~\cite{epos-lhc}. 
Together with the \texttt{iact/atmo} extension discussed in the next section,
these are the major changes from using \gls{mmcs} \texttt{6.5} to \gls{corsika} \texttt{7.69}.
The most advanced model is currently \texttt{EPOS-LHC}~\cite{epos-lhc}, which combines the older
\texttt{VENUS} and \gls{qgsjet} models with better extrapolation and \gls{LHC} data.


\subsection{Cherenkov Light Production}\label{sec:corsika-cherenkov}

Any charged particle traveling faster than the local speed of light induces
emission of Cherenkov light.
\gls{corsika} divides the tracks of particles into steps between interactions,
decays or the deflection in Earth's magnetic field.
These track parts along with start and end energy of the particle are
handed over to the subroutines calculating the Cherenkov emission.

A first version of the Cherenkov production code, used by the \gls{mmcs} \texttt{6.5}
production, was developed for non-imaging detectors of the \gls{hegra} array and proofed
to be not ideal for \glspl{iact}.
The later developed \texttt{iact/atmo} extension improves several aspects:
photons are collected in spheres encompassing the telescopes instead of
rectangles on the ground,
better description of the atmosphere by using fine-grained, tabulated values
for density and refractive index and taking atmospheric refraction of the Cherenkov
photons into account~\cite{iactatmo}.
Due to these improvements, the \texttt{iact/atmo} extension will be used for the
new simulations using \gls{corsika} \texttt{7.69}.
The \texttt{iact/atmo} extension writes its output in the \texttt{eventio} format~\cite{eventio},
a generic container format for binary payloads,
which are defined by the applications writing and reading the data.
Two different formats exist for the Cherenkov payload,
a smaller representation called \enquote{compact} 
with limited precision using 16-bit integers to store scaled quantities and a larger
storing 32-bit floating point values.

Both the \texttt{iact/atmo} extension and \gls{mmcs} offer the possibility
to \emph{reuse} a shower.
Meaning that the same telescope is randomly placed at different 
positions in the Cherenkov light pool of a shower,
effectively creating multiple events from the same air shower.
This results in a very direct speed up of simulations and since
most of the simulated showers will not trigger the telescope or survive
the event selection in the analysis, it is quite rare that the same shower 
is used multiple times in the end, which could introduce a statistical
bias as events would not be independent anymore.
Because computation time and small event numbers are mainly a problem of 
hadronic simulations, a reuse of 20 is only used for these showers and not for photon primaries.
Another option of improving simulation performance is treating several photons
as a single ray with a weight corresponding to the number of photons.
This is called bunching and is a compromise between accuracy of the spatial 
distribution and computing time.

Because \gls{fact}'s photo detectors are sensitive to larger wavelengths then
classical \glspl{pmt}, Cherenkov photons are simulated in wavelengths from \SIrange{290}{900}{\nano\meter}.  

\subsection{Particle Properties and Other Configuration Options}
All runtime options for an individual corsika run, meaning a single execution of
the program simulating a fixed number of showers, are configured in a 
plain text file called the input card.
This includes properties of the primary particles, 
description of the telescopes, seeds for the random generator to ensure
reproducibility, the local magnetic field, the atmosphere and many more.

The initial parameters of any particle in the simulation are
\begin{itemize}[nosep]
  \item Particle type
  \item Energy
  \item Direction
  \item Position, expressed as the impact point on the ground for primaries.
\end{itemize}
At the beginning of each new simulation event, these properties are either
fixed or drawn randomly from the corresponding distributions.
In the simulations done for \gls{fact}, the particle type
is always kept fixed, either gamma rays, protons or helium nuclei are simulated.
The energies of the simulated events are drawn from a power law distribution with
spectral index $\gamma$ between a minimum ($\Emin$) and a maximum energy ($\Emax$).
The distance to the telescope $R$, also known as the impact parameter,
is drawn so that the events are inside a circle of radius $R_{\max}$ and uniform in area.
The direction is sampled uniformly in the azimuth angle $φ$ between $φ_{\min}$ and
$φ_{\max}$
and the zenith angle $ϑ$ is sampled to be uniform in solid angle between $ϑ_{\min}$
and $ϑ_{\max}$.
The corresponding probability density functions are given in Appendix~\ref{apx:corsika-pdfs}.

\section{Simulation of Detector Response using CERES}\label{sec:ceres}

The detector simulation of \gls{fact} is done using \ceres{}~\cite{ceres}, an executable
using \gls{fact}'s version of the \gls{mars}\footnote{While sharing a common history, developement of \gls{fact}'s and \gls{magic}'s version of the software split before \gls{fact} started operation in 2009. \gls{magic}'s version is not public.} framework to simulate the detector response
given Cherenkov photons on the ground as input.
\ceres{} supports both the classical \gls{corsika} binary output format and the \texttt{eventio}
output format produced by the \texttt{iact/atmo} extension as discussed in the \hyperref[sec:corsika]{previous section}. 
Reading of \texttt{eventio} however is rudimentary and does not support all features
of \gls{corsika}, e.\,g.\ only a bunchsize of 1 and the compact format of the \gls{corsika}
output are supported.
As with \gls{corsika}, \ceres{} has a multitude of configuration options,
describing the telescope, the pointing position relative to the shower axis and the environmental conditions.

The first step in the simulation in \ceres{} is the application of several absorption
processes, reducing the amount of photons the simulation needs to process.
As each photon that finally might be detected in one of the telescope's pixels 
goes through the same processes that might absorb the photon, all of these
processes can be combined into an overall detection efficiency, greatly reducing 
computational cost through applying this absorption as early as possible.
These absorptions include atmospheric scattering, reflectivity of the mirrors,
transmittance of the entrance window of the camera and the Winston cones and
the photon detection efficiency of the \glspl{sipm}.
All these efficiencies are tabulated vs.\ wavelength in files as input to the simulation.
For the simulations produced for \cite{phd-temme}, an additional absorption without
direct physical cause further reducing the number of photons was introduced.
This is used to model all neglected sources of photon loss, like shadowing effects
of the mast structure holding the camera.

After discarding photons according to the overall absorption,
simplified ray tracing is done to simulate the optical system,
calculating which pixel if any is hit by the photon.
The segmented mirror is defined in a configuration file,
providing shape, size, focal length, position and orientation of each facet.

The signal of the \glspl{sipm} is calculated by superimposing the pulse shape
of a single photon for each individual photon, also adding additional photons
for the \gls{nsb}, dark counts and cross talk.
The pulse shape can either be defined using tabulated values in a text file
or an analytical function.
An additionial electronic noise, only simulated as simple white noise is added to the resulting signal.

Several options to further add artifacts to the simulated timeseries 
were implemented for the simulations produced for \cite{phd-temme}.
It was observed that the temporal features are not well-described, 
especially the variance of photon arrival times was much smaller for simulated
events than observed on measured data.
Three different approaches were introduced to tackle this problem:
a fixed time offset for each pixel, so that the signals would go out of sync,
a random offset between pixels for each event and smearing the arrival times
of each individual photon by a normally distributed random number.

The signal is then used to simulate the trigger decision and if any trigger
is raised, the event is stored to disk.
\ceres{} supports both \texttt{ROOT} output as well as the \gls{fits} format,
which produces files very similar to the ones the actual telescope data acquisition
writes.

The \gls{mars}/\ceres{} version used for the simulations done in this work is revision 19561
in \gls{fact}'s subversion repository~\cite{fact-svn}.


\section{Reweighting Simulated Events to Physical Fluxes}

To compare observed event distributions to simulated ones,
to calculate sensitivities for a given gamma-ray source or
to fit flux models to observed events, a simulated
event distribution needs to be reweighted to represent a physical flux.

The first step is to calculate the flux normalization of the simulated events $\Phi_\text{sim}$
for a given observation time \tobs.
The number of simulated events $N$ is the integral of the differential flux:
\begin{align}
  N &= \int_{\Emin}^{\Emax} \int_0^{\tobs} \int_Ω \int_A \Phi_{\text{sim}} \left(\frac{E}{\Eref}\right)^γ 
    = \frac{\Phi_\text{sim}\tobs Ω A}{\Eref[γ] \, (γ + 1)} \left(\Emax[γ + 1] - \Emin[γ + 1]\right) .\\
    \intertext{Solving for $\Phi_\text{sim}$ yields}
  \iff  \Phi_\text{sim} &= \frac{\Eref[γ] \, (γ + 1) N}{\tobs Ω A \left(\Emax[γ + 1] - \Emin[γ + 1]\right)}
\end{align}
Where $A = 2 \PI \Rmax[2]$ and $Ω$ is the solid angle, which is given by
\begin{equation}
  Ω = 2 \PI (1 - \cos{\theta_{\max}}) \, \si{\steradian}
\end{equation}
if the events were simulated using a viewcone and $Ω = 1$ if a point-like source
was simulated.

The sample weights for each simulated event are now calculated as the ratio 
of the target energy spectrum and simulated energy spectrum
\begin{equation}
  w_i = \frac{\Phi_\text{target}(E_i)}{\Phi_\text{sim}(E_i)} \label{eq:weights}
\end{equation}


\section{Monte Carlo Production Software}\label{sec:mopro}


To run the large number of tasks required for the Monte Carlo simulations,
a software library and a set of executable programs were developed,
with the following requirements identified:

\begin{itemize}[noitemsep]
  \item Templating for the configuration files of the simulation programs,
    so that job parameters can be stored centrally.
  \item Tracking of job status, input and output files.
  \item Automatic installation and compilation of the needed software.
  \item With the above, also the possibility to use several compiled versions
    of \gls{corsika} and \texttt{ceres} to enable comparisons.
  \item Possibility to run computation tasks on multiple backends,
    at least on the \texttt{SLURM}~\cite{slurm} cluster engine and locally using multiple CPU cores.
  \item Reproducibility of jobs.
  \item Central job queue optionally with multiple consumers.
  \item Collection and compression of simulation results.
  \item Possibility to use multiple database engines, at least
    \texttt{sqlite} for easy, local processing and testing and a
    fully grown relational database like MySQL or PostgreSQL for larger
    productions.
\end{itemize}

The software called \mopro~\cite{mopro3} was implemented in \texttt{python}.
Two similar projects existed before, but were centered around the
older \gls{mmcs} and had several shortcomings with respect to the requirements
listed above, lacking support for automatic installation of software,
multiple consumers, running jobs locally and on multiple sites.

The job configuration and simulation settings are stored in a relational
database, using the \texttt{peewee}\cite{peewee} Object Relational Mapper.
The relational mapper allows access to the database from \texttt{python} as objects
and supports all major database systems.

Templating of the configuration files is done using the \texttt{jinja2} templating
library, an example of a \gls{corsika} input card template can be found in Appendix~\ref{apx:corsika_input}, the \ceres{} template in Appendix~\ref{apx:ceres-template}.
The expressions in double-curly braces are evaluated and replaced for each
database entry before giving it as input to \gls{corsika} and \ceres{}, respectively.

Two computation backends are currently implemented.
First, jobs can  be submitted
on a cluster using \texttt{SLURM}, which is used for job management
at the LIDO3 cluster in Dortmund and the LESTA cluster in Geneva, where \gls{fact}'s data
is stored.
Jobs can also be processed using local worker processes, which eases development
and running small simulation productions.
Other backends can be added by implementing a small number of required methods, e.\,g.\ for submitting, canceling and getting the status of jobs.

The main program queries the database in certain intervals for pending jobs,
and if possible submits them for computation to the workers.
An extra thread in the main program uses a \texttt{ØMQ}~\cite{zeroMQ} server socket
to receive job status updates which are then persisted to the database.

The simulation programs \gls{corsika} and \ceres{} are wrapped in small scripts to
copy input files to the local working node, handle failures and communicate
status transitions to the main program as mentioned above.

When a new job is submitted to the workers,
the main program makes sure all necessary software is installed, creates
the required directories for output and log files and
creates the configuration files from the templates and parameters stored
in the database.

As any \ceres{} job depends on the successful \gls{corsika} job it uses 
as input, \ceres{} jobs are started when their corresponding \gls{corsika}
job finishes successfully and it was computed at the same location.

\texttt{mopro3} itself is configured using a \texttt{yaml} file.
The settings include credentials for downloading \gls{corsika},
the database configuration, 
which temporary directory to use for job processing
and options for the workers like which queues to use on a \texttt{SLURM} cluster
or how many CPUs to use in parallel for local processing.

\subsection{Database Structure}

The \mopro{} database contains five tables,
one describing general settings for \gls{corsika},
one for each individual \gls{corsika} run, the same two tables for the
\ceres{} jobs and one table describing the job states.

For the \gls{corsika} settings and runs, a compromise between flexibility and
complexity of the database and implementation was made.
\gls{corsika} has both compile time and execution time options.
The compile time options are normally set interactively by executing
the \gls{coconut} program.
This creates a header file \texttt{config.h} defining preprocessor variables that result in conditionally
compiling the corresponding code. 
This is used among other things for the interaction models, thus the same \gls{corsika}
installation cannot be used with different interaction models.
Instead of implementing all possible compile time options in
the \texttt{CorsikaSettings} table and generating the \texttt{config.h} file
at installation time, the user has to generate the file using \gls{coconut} and 
the full file is stored in the database and used to compile \gls{corsika}.
This also has the advantage of being rather agnostic about the version of \gls{corsika}.

A similar approach is taken for the large number of runtime 
configuration options \gls{corsika} supports, which in turn also depend on
the compile time configuration.
Only the most fundamental settings most likely to change
between runs of the same kind are stored in the database.
Other parameters have to be changed via the input card template in the \texttt{CorsikaSettings} table.
This also has the advantage that they are very unlikely to change with a new release of \gls{corsika}, as all these options are very fundamental,
improving maintainability of \texttt{mopro3}.
In most cases, no changes should be required when a new minor version update of \gls{corsika} is released.

The \texttt{Status} table stores all possible states a run can reside in.
All states and their possible state changes are shown in \autoref{fig:states}.
\begin{figure}
  \centering
  \begin{tikzpicture}
    \tikzset{change/.style={font=\footnotesize, text width=3cm, color=gray, midway}}
    \node[fill=blue!20, font=\ttfamily] (created) {\strut created};
    \node[fill=blue!20, font=\ttfamily, right=1cm of created.east] (queued) {\strut queued};
    \node[fill=blue!20, font=\ttfamily, right=1cm of queued.east] (running) {\strut running};
    \node[fill=green!20, font=\ttfamily, right=3cm of running.east] (success) {\strut success};
    \node[fill=red!20, font=\ttfamily, below right=1cm and 3cm of running.east] (failed) {\strut failed};
    \node[fill=red!20, font=\ttfamily, above right=1cm and 3cm of running.east] (walltime) {\strut walltime\_exceeded};

    \draw[thick, ->] (created.east) to[out=30, in=150] node[change, above=0.25cm, align=center] {Job is submitted to a cluster} (queued.west);

    \draw[thick, ->] (queued.east) to[out=-30, in=210] node[change, below=0.25cm, align=center] {Job starts executing} (running.west);

    \draw[thick, ->] (running.east) to node[change, above, align=center] {Job finished} (success.west);

    \draw[thick, ->] (running.east) to[out=-45, in=180] node[change, below left, align=center] {An error occurred} (failed.west);

    \draw[thick, ->] (running.east) to[out=45, in=180] node[change, above left, align=right] {Job not finished in walltime} (walltime.west);
  \end{tikzpicture}
  \caption{%
    Possible state transitions for the \texttt{mopro3} processing.
    Blue indicates unfinished jobs, green successfully finished jobs and red failure modes.
  }\label{fig:states}
\end{figure}



\section{Comparison of FLUKA and URQMD}

\begin{figure}
  \centering
  \begin{subfigure}{0.5\textwidth}%
    \includegraphics[page=1, width=\linewidth]{build/plots/fluka_urqmd_comparison.pdf}%
  \end{subfigure}%
  \begin{subfigure}{0.5\textwidth}%
    \includegraphics[page=2, width=\linewidth]{build/plots/fluka_urqmd_comparison.pdf}%
  \end{subfigure}\\
  \begin{subfigure}{0.5\textwidth}%
    \includegraphics[page=3, width=\linewidth]{build/plots/fluka_urqmd_comparison.pdf}%
  \end{subfigure}%
  \begin{subfigure}{0.5\textwidth}%
    \includegraphics[page=4, width=\linewidth]{build/plots/fluka_urqmd_comparison.pdf}%
  \end{subfigure}%
  \caption{%
    Image parameters \param{concentration\_cog}, \param{length}, \param{size} and \param{width}  for
    the three different simulation data sets. 
    The differences between \gls{urqmd} and \gls{fluka} are negligible while
    there is a slight difference between the two and the older simulation settings,
    e.\,g.\ for large values of \param{size} and small values of \param{concentration\_cog}.
  }\label{fig:fluka_urqmd}
\end{figure}

As described in section \ref{sec:models}, there are two options to consider for the
low energy hadronic interaction model.
To compare the two, a smaller test data set using protons as primary particle was produced for each \gls{fluka} and \gls{urqmd}. 

The full simulation and analysis chain up to the image parameters as described in
\autoref{chp:preprocessing} was executed.
\autoref{fig:fluka_urqmd} shows the distributions of several image parameters
for the two new simulation sets and the existing set produced with \gls{mmcs} \texttt{6.5}.
While the new version of \gls{corsika} shows some differences compared to
the older version, the differences between the two interaction models are negligible 
on the image parameter level.
Considering the problematic license of \gls{fluka} and that \gls{urqmd} also runs
simulations about twice as fast, it was decided to use \gls{urqmd} for the new production
of \gls{fact} simulations.
